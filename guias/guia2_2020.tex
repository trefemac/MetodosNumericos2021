\documentclass[a4paper]{article}
\usepackage[latin1]{inputenc}
%\usepackage{babel}
\topmargin -.7truecm
\textheight 23.5truecm
\textwidth 16.5truecm
\oddsidemargin 0.2truecm
\parskip 1truemm
%\parsep 1truemm
\parindent 0em
\usepackage{amssymb}

\def\d{\,{\rm d}}

\def\RR{{\mathbb R}}
\def\CC{{\mathbb C}}
\def\NN{{\mathbb N}}
\def\ZZ{{\mathbb Z}}
\def\EE{{\mathbb E}}
\newcounter{problema}
\newcommand{\prob}[1]{\vspace{0.33cm}\stepcounter{problema}
                 \noindent{\bf Problema \arabic{problema}:}{\it #1}}

\begin{document}
\renewcommand{\labelenumi}{\bf \alph{enumi})}
\renewcommand{\labelenumii}{\alph{enumi}$_\arabic{enumii}$)}
\thispagestyle{empty}
\begin{center}
{\large {\bf M\'etodos Num\'ericos}}\\
{\small Aula Virtual: http://www.famaf.proed.unc.edu.ar/course/view.php?id=451} \\
%{\small http://www.famaf.unc.edu.ar/$\sim$ftamarit/Numerico\_2017/} \\

\rule{0em}{2.em}{\bf Gu\'{i}a 2} \\ Abril de 2020
\end{center}

% p1
\prob{}
Cree un directorio dentro de su directorio metnum, con nombre guia2. Luego copie el archivo $guia02$\_$2019.pdf$  
que se encuentra dentro del directorio /tmp,  al directorio creado.

		{\em Comentario: todo archivo que se cree dentro del /tmp (temporal) dura un tiempo finito. 
En general se setea el sistema para que borre todo lo que est\'a dentro del temporal, cada vez que se reinicia 
la m\'aquina. Conclusi\'on, c\'opielo ahora porque la pr\'oxima clase no estar\'a ahi.}



% p
\prob{} Use aritm\'etica de 4 d\'igitos (redondeando) para simular el problema del c\'alculo computacional
de $\pi - \frac{22}{7}$. Luego calcule el error absoluto y el error relativo de la representci\'on de $\pi$ y de $\frac{22}{7}$,
y el error relativo de la diferencia.

% p2
\prob{} Interprete el resultado de los siguiente programas en virtud de la representaci\'on
de punto flotante de los n\'umeros reales.\\
\begin{itemize}
\item 
{\fontfamily{phv}\selectfont
\hspace*{4cm}  program test\textunderscore igualdad  \\
\hspace*{4cm}  implicit none  \\
\hspace*{4cm}  integer, parameter   \hspace*{0.5cm}  :: pr=kind(1.d0)  \hspace*{0.5 cm} ! pr puede ser simple o doble   \\ 

\hspace*{4.2cm}  if (19.08\textunderscore pr + 2.01\textunderscore pr == 21.09\textunderscore pr) then  \\
\hspace*{4.5cm}     write(*,*) '19.08 + 2.01 = 21.09 '  \\
\hspace*{4.4cm}  else  \\
\hspace*{4.6cm}     write(*,*) '19.08 + 2.01 /= 21.09 '  \\
\hspace*{4.2cm}  endif  \\
\hspace*{4cm}  end program test\textunderscore igualdad }  \\

\item 
{\fontfamily{phv}\selectfont
\hspace*{4cm}  program test\textunderscore igualdad2  \\
\hspace*{4cm}  implicit none  \\
\hspace*{4cm}  integer, parameter   \hspace*{0.5cm}  :: pr=kind(1.d0)  \hspace*{0.5 cm} ! pr puede ser simple o doble   \\  
\hspace*{4cm}  real(pr)             \hspace*{0.75cm} :: a  \\ 

\hspace*{4.2cm}  a = 2.05\textunderscore pr  \\
\hspace*{4.2cm}  if (a*100.\textunderscore pr  == 205.\textunderscore pr) then  \\
\hspace*{4.5cm}     write(*,*) '2.05*100 = 205 '  \\
\hspace*{4.4cm}  else  \\
\hspace*{4.6cm}     write(*,*) '2.05*100 /= 205 '  \\
\hspace*{4.2cm}  endif  \\
\hspace*{4cm}  end program test\textunderscore igualdad2 }  \\

\end{itemize}
Piense el mensaje de este ejercicio, el cual debe tener presente en toda la materia.\\

\prob{} Escriba un programa que calcule $\epsilon_m$ de su m\'aquina, en
simple y doble precisi\'on.

\newpage

\prob{} De ejemplos para mostrar que la matem\'atica de punto flotante
\begin{enumerate}
\item no es cerrada respecto a la suma ni a la multiplicaci\'on, 
\item no es asociativa respecto a la suma ni a la multiplicaci\'on,
\item la multiplicaci\'on no es  distributiva respecto a la suma.  
\end{enumerate} 
Haga el programa Fortran correspondiente.

%

\prob{} Escriba un programa que comprueba las reglas de 
la matem\'atica de no detenci\'on con {\tt $\pm$Infinity} , {\tt $\pm 0$} y
{\tt NaN}. Luego, compile el mismo programa con la {\em opci\'on} 
{\tt -ffpe-trap=zero,invalid,overflow} y compare los resultados.

Nota: puede crear un {\em alias} en su archivo {\em .bashrc} para utilizar
siempre el gfortran con esta opci\'on.


\prob{} Escriba el siguiente programa y explique por qu\'e los valores
obtenidos no son iguales: \\ 

{\fontfamily{phv}\selectfont
\hspace*{4cm}   program test  \\
\hspace*{4cm}   implicit none  \\
\hspace*{4cm}   integer, parameter   \hspace*{0.1cm}  :: pr = kind(1.0)\\
\hspace*{4cm}   real(pr)         \hspace*{1.8cm}       :: sum0,sum1 \\
\hspace*{4cm}   integer          \hspace*{1.9cm}     :: i  \\
\hspace*{4cm}   ! \\
\hspace*{4cm}   sum0 = 0.\_pr ; sum1 = 1.\_pr \\
\hspace*{4cm}   do i=1,10000 \\
\hspace*{4cm}   \hspace{.3cm} sum0 = sum0 + 1.e-8\_pr \\
\hspace*{4cm}   \hspace{.3cm} sum1 = sum1 + 1.e-8\_pr \\ 
\hspace*{4cm}   end do \\
\hspace*{4cm}   sum0 = sum0 + 1.\_pr  \\
\hspace*{4cm}   write(*,*) sum0, sum1  \\
\hspace*{4cm}   end program test }


% p3
\prob{} Implementar un programa Fortran para evaluar la suma (en precisi\'on
simple)
\[
\sum_{n=1}^{10\,000\,000} \frac{1}{n}
\]
\noindent
primero, en el orden usual, y luego, en el orden opuesto. Explique las diferencias
obtenidas  e indique cu\'al es m\'as preciso y su justificaci\'on.

% 3
\prob{} Efect\'ue con un programa en Fortran en simple precisi\'on los siguientes
c\'alculos, matem\'aticamente equivalentes,

\begin{enumerate}
\item $1\,000\,000 \times 0.1$

\item $\sum_{n=1}^{1\,000\,000} 0.1$

\item $\sum_{m=1}^{1\,000} \left( \sum_{n=1}^{1\,000} 0.1\right)$


\end{enumerate}
\noindent

{ Explique las diferencias obtenidas entre resultados finales de a), b) y c) y muestre que el error relativo en b) es del orden
	del 1\%, pero es mucho menor en c). Resalte la conclusi\'on de este ejercicio. }


\begin{enumerate}
\item[{\bf d)}] En los puntos b) y c), vaya guardando con un write (cada 1000 iteraciones en el caso b, 
y en todas las iteraciones de la suma externa en c), los errores parciales en un archivo de datos.  
La primer columna con los valores de los \'indices (multiplicados por 1000 para el caso c) de las sumas 
y segunda columna con el error parcial, i.e., la diferencia entre el valor de la respectiva suma parcial y el valor exacto ($i*0.1$).  
Grafique usando {\em gnuplot}, los errores parciales de b) y c) superpuestas, en funci\'on del 
\'indice correspondiente, en un archivo``prob9.ps" en color, con t\'itulo de gr\'afica``Problema 9, 
Gu\'ia 2", con nombres adecuados en los ejes y con leyendas adecuadas. 
Los datos del b), graficarla con puntos de tama\~no 0.75 (el default es 1), y los de c) con l\'inea
de grosor 1.6 (el default es tambien 1). 
Luego de ver c\'omo le queda la gr\'afica en escala lineal, cambie el eje x, 
a escala logar\'itmica. Seg\'un su criterio, en qu\'e escala se 
aprecia mejor el resultado del problema para su an\'alisis? Puede al eje $y$ darle escala 
logar\'itmica? Pruebe qu\'e pasa si lo hace.
Analice la conveniencia en general del uso de escalas logar\'itmicas.

Finalmente deje el archivo grabado en escala lineal en $y$, y logar\'itmica en $x$, y diga el 
tama\~no del mismo tanto en bytes como Kbytes. 

Si prefiere a usar otro graficador, se incentiva a que lo usen, siempre y cuando logren el 
archivo final con mismos requisitos que le pedimos.


\end{enumerate}
 

% 
\prob{} Sup\'ongase que $x$ e $y$ son n\'umeros positivos correctamente redondeados
a $t$ d\'igitos. Mostrar que la magnitud del error relativo de redondeo de
$z=x-y$ est\'a acotada por
$$
\left| \frac{\Delta z}{z} \right| \leq \frac{|x| + |y|}{|x-y|}\ {\bf u} + {\bf u},
$$
\noindent
donde ${\bf u}$ es la unidad de redondeo,  ${\bf u}=\frac12 \epsilon_M$


% 4
\prob{} La f\'ormula cuadr\'atica nos dice que las ra\'ices de $ax^2 + bx + c = 0$
son
$$
x_1 = \frac{-b + \sqrt{b^2 -4ac}}{2a}, \qquad \qquad x_2 =  \frac{-b - \sqrt{b^2 -4ac}}{2a} .
$$
\noindent
Si $b^2\gg 4ac$, entonces, cuando $b>0$ el c\'alculo de $x_1$ involucra en el
numerador la sustracci\'on de dos n\'umeros casi iguales, mientras que si $b<0$,
esta situaci\'on ocurre para el c\'alculo de $x_2$. ``Racionalizando el numerador''
se obtienen las siguientes f\'ormulas alternativas que no sufren este problema:
$$
x_1 = \frac{-2c}{b + \sqrt{b^2 -4ac}}, \qquad \qquad x_2 =  \frac{2c}{-b + \sqrt{b^2 -4ac}} ,
$$
\noindent
siendo la primera adecuada cuando $b>0$, y la segunda cuando $b<0$.
Escriba un programa en precisi\'on simple que utilice la f\'ormula usual y la 
``racionalizada'' para calcular las ra\'ices de
$$
x^2 + 6210 x + 1 = 0.
$$
\noindent 
Interprete los resultados.

\bigskip



%6 
\prob{ Problema matem\'aticamente inestable. } Considere la sucesi\'on 
\begin{equation}
\label{recurrencia}
x_n\,=\,\frac{13}{3} \,x_{n-1} - \frac{4}{3}\, x_{n-2} \,.
\end{equation}

\begin{enumerate}
\item Demuestre que, eligiendo $x_0=1\;,x_1=1/3$ tenemos que  $x_n=1/3^n\;\forall n\geq 0$ 
(sugerencia: use inducci\'on).
\item Haga un c\'odigo fortran que calcule  $x_n$ y su error relativo hasta $n=15$ y discuta
el resultado comparando reales de 4 y 8 bytes. 
\item Defina $y_n=1/x_n$ y encuentre la relaci\'on de recurrencia para  $y_n$. Imponga la 
condici\'on inicial $y_0=1, \;y_1=3$. Calcule ahora $x_n=1/y_n$ y compare con lo obtenido en el 
punto anterior. Es este algoritmo estable? discuta.
\item Verifique que la soluci\'on general de la ecuaci\'on (\ref{recurrencia}) con $x_0,\,x_1$ 
arbitrarios es
\[
x_n\,=\,\frac{A}{3^n} \,+\,B\,4^n\, .
\]
\noindent
Note que los valores iniciales elegidos en a) y b) corresponden al
caso particular $A=1$ y $B=0$. Discuta en base a esto los resultados num\'ericos obtenidos.
\end{enumerate}

\bigskip




\noindent {\bf Ejercicios Complementarios}


%
\prob{} Suponga que los puntos $(x_0,y_0)$ y $(x_1,y_1)$ est\'an en una l\'inea recta, con $y_1=/y_0$.
Se dispone de dos f\'ormulas para obtener el valor $x^*$ de la intersecci\'on de la recta con el eje {\em x}:
\[
x^* = \frac{x_0y_1 - x_1 y_0}{y_1-y_0} \qquad \qquad y  \qquad \qquad
x^* = x_0 - \frac{(x_1-x_0) y_0}{y_1-y_0}
\]
\begin{enumerate}
\item Muestre que ambas f\'ormulas son algebraicamente correctas
\item Use los datos $(x_0,y_0) = (1.31,3.24)$ y $(x_1,y_1) = (1.93,4.76)$ y aritm\'etica de tres d\'igitos
para calcular $x^*$ de ambas formas.  Cu\'al m\'etodo es mejor? Por qu\'e'? 
\end{enumerate}

%5
\prob{} Considere las siguientes integrales
\[
y_n = \int_0^1 \frac{x^n}{x+10}dx
\]
\noindent 
para $n=1,2,\dots ,30$. Muestre que
\[
y_n = \frac{1}{n} - 10 y_{n-1} \ ,
\]
\noindent
y que  $y_0 = \ln(11) - \ln(10)$.
Note que empleando esta f\'ormula de recursi\'on, se obtienen los
resultados exactos de las integrales.  

\begin{enumerate}
\item Escriba un programa en precisi\'on simple que a partir
de $y_0$, calcule recursivamente $y_i$ para $i=2,\cdots ,30$. 
Explique los resultados obtenidos (note que $0 < y_n < 1$).

\item Derive una f\'ormula para evaluar $y_{n-1}$ dado
$y_n$. Escriba un programa que utilice esta recursi\'on
para calcular $y_n$, aproximando $y_{n+k}$ por 0.
Explique por qu\'e este algoritmo es estable. Encuentre
el valor de $k$, para que el programa calcule $y_{7}$ con un error 
absoluto menor a $10^{-6}$ (note que $y_7 \approx 0.0114806$).

\item Modifique el programa para que tome como entrada $n$, y el
error absoluto deseado, $\epsilon$, y luego estimando el error absoluto
en el calculo de $y_n$ como $Err = |\hat{y}_n(y_{n+k}=0) - \hat{y}_n(y_{n+k-1}=0)|$
($\hat{y}_n(y_{n+k}=0)$ es el valor de $y_n$ obtenido partiendo de $y_{n+k}=0)$),
determine $y_n$ con un error absoluto (aproximado) menor que $\epsilon$.

\end{enumerate}

\end{document}
