\documentclass[a4paper]{article}
\usepackage[latin1]{inputenc}
\usepackage{amsmath}
%\usepackage{babel}
\topmargin 0truecm
\textheight 23truecm
\textwidth 16.5truecm
\oddsidemargin 0truecm
\parskip 1truemm
%\parsep 1truemm
\parindent 0em
\usepackage{amssymb}

\def\d{\,{\rm d}}

\def\RR{{\mathbb R}}
\def\CC{{\mathbb C}}
\def\NN{{\mathbb N}}
\def\ZZ{{\mathbb Z}}
\def\EE{{\mathbb E}}
\newcounter{problema}
\newcommand{\prob}[1]{\vspace{0.33cm}\stepcounter{problema}
                 \noindent{\bf Problema \arabic{problema}:}{\it #1}}

\begin{document}
\renewcommand{\labelenumi}{\bf \alph{enumi})}
\renewcommand{\labelenumii}{\alph{enumi}$_\arabic{enumii}$)}
\thispagestyle{empty}
\begin{center}
{\large {\bf M\'etodos Num\'ericos}}\\
{\small Aula Virtual: http://www.famaf.proed.unc.edu.ar/course/view.php?id=547} \\

\vspace*{1.5cm} 
\rule{0em}{2.em}{\bf Gu\'{i}a 1} \\ Marzo de 2020
\end{center}
\vspace*{.5cm} 

\prob{} En una terminal, cree un directorio nuevo. Descargue desde el
navegador el archivo de esta gu\'ia y mueva el mismo al nuevo directorio.
Cambie el nombre del archivo a {\em guia1.pdf} y copie el archivo a otro 
con nombre {\em Copia.pdf}.  Luego, liste el contenido del
directorio y verifique los nombres y el tama\~no y fecha de los mismos.
Averig\"ue qu\'e espacio ocupa el directorio usando el comando {\em du -h}.
Finalmente, borre todos los archivos y el directorio.

\prob{} Utilice el programa {\em gnuplot} para graficar la funci\'on
$f(x) = \sin (2\pi x)$, para $x  \in [0:5]$. Ajuste la cantidad de puntos
que usa gnuplot para graficar ({\em samples}) a 500 para mejorar la calidad
del gr\'afico. De nombre a los ejes, t\'itulo al gr\'afico y defina la
leyenda para que aparezca $f(x)$.

\prob{} Abra una terminal y realice las siguientes acciones:
\begin{enumerate}
\item Cree un directorio con el nombre {\em MetNum}

\item En ese nuevo directorio ejecute el editor {\em kate} liberando
la l�nea de comandos (utilice el s�mbolo $\&$ o suspenda el proceso con 
{\em Ctrl + Z} y luego ingrese {\em bg}).

\item En el editor de texto ingrese los siguientes datos en tres columnas,
separados por dos espacios:
\begin{center}
\begin{tabular}{ccc}
0.0  & 1.0  & 1.2\\
1.0  & 1.5  & 1.6  \\
2.0  & 2.0  & 2.0 \\
3.0  & 2.5  & 2.4 \\  
\end{tabular}
\end{center}
\noindent
Grabe esta tabla en el archivo {\em datos.dat}. Aseg\'urese que est\'e
en el directorio {\em MetNum}. No cierre el editor. 

\item Ahora utilice {\em gnuplot} para graficar la 2da y la 3er columna versus
la primera. Grafique usando solo puntos. En el mismo gr\'afico incluya las funciones 
$y = 0.5 x + 1$ y $y=0.4 x + 1.2$. Nombre al eje horizontal {\em x} y al vertical {\em y}. Habilite la grilla. Titule el gr\'afico {\em Funciones lineales}. Cree archivos
de salida en formato postscript color y blanco y negro (elija nombres con 
terminaci\'on .ps). Abra otra terminal y con la aplicaci\'on {\em  okular} visualice
los archivos de salida, a medida que los va generando). Lluego, experimente graficar 
cambiando el tama\~no de los puntos y el grosor de las l\'ineas, etc. Modique el archivo en la ventana de {\em kate} (que est\'a abierta) y actualice el gr\'afico para constatar los cambios. Finalmente, cierre {\em gnuplot}.

%\item Realice un gr\'afico similar usando la aplicaci\'on {\em xmgrace}. Explore las distintas alternativas que ofrece esta apliaci\'on. 

\item Liste los archivos del directorio mostrando su tama\~no y fecha. Averig�e qu\'e
espacio ocupa el directorio {\em MetNum}. Finalmente, cierre la ventana del editor. 
Borre los archivos postscript y cierre la terminal.
\end{enumerate}


\newpage
% 
\prob{} Evaluar (en papel y l\'apiz) las siguientes expresiones, anticipando el
resultado de estas operaciones en Fortran 90. Verifique programando las mismas en Fortran 90.
\begin{enumerate}
\item {\tt 5 / 2 + 20/ 6} 
\item {\tt 4 * 6 / 2 - 15 / 2}
\item {\tt 5 * 15 / 2 / (4 - 2)} 
\item {\tt 1 + 1/4}
\item {\tt 1. + 1/4}
\item {\tt 1 + 1./4}
\item {\tt 1. + 1./4.}
\end{enumerate}

\prob{}   Evaluar (en papel y l\'apiz) las siguientes operaciones
usando matem\'atica exacta y {\em matem\'atica  de enteros de 2 bytes} (Integer(2)). 
Verifique sus resultados escribiendo un c\'odigo en Fortran 90. 
\begin{enumerate}
\item {\tt 32767 + 1}
\item {\tt 30000*2}  
\item {\tt -30000 - 10000}  
\end{enumerate}% 2
\prob{} Escriba un programa que pida dos n\'umeros reales e imprima en la
pantalla el mayor de ellos. El programa debe indicar si los n\'umeros son iguales.

% 3
\prob{} Escriba un programa que pida un n\'umero entero y determine si es
m\'ultiplo de 2 y de 5.

% 4
\prob{} Escriba una programa que ingrese los coeficientes $A$, $B$ y $C$ de un
polinomio real de segundo grado ($A x^2 + B x + C$), calcule e imprima en
pantalla las dos ra\'ices del polinomio en formato complejo $a + i b$, sin
utilizar algebra compleja.


\prob{} Escriba un programa que permita convertir n\'umeros naturales con base 
$10$ a la base $b <= 16$. El programa
debe pedir como entrada $b$ y el n\'umero natural a convertir (que debe estar en
base 10). 

% 5
%\prob{} Dise\~nar una funci\'on que calcule la potencia en\'esima de un n\'umero, es
%decir que devuelva $X^n$ para $X$ real y $n$ entero. Realice un programa que
%utilice la funci\'on e imprima en pantalla las primeras 5 potencias naturales de
%un n\'umero ingresado.


%\prob{} Los per\'{\i}metros $\pi_k$ de poligonos regulares de $k$ lados inscriptos en un 
%c\'{\i}rculo de diametro unidad son $\pi_8=3.061467,\,\pi_{16}=3.121445,\,\pi_{32}=3.136548,\,\pi_{64}=3.140331$ 
%respectivamente, mientras que los per\'{\i}metros de poligonos regulares circunscriptos% son 
% $\Pi_8=3.313708,\,\Pi_{16}=3.182598,\,\Pi_{32}=3.151725,\,\Pi_{64}=3.144118$
%respectivamente.  Asuma que 
%
%\[
%\pi_k\,=\,c_0+\frac{c_1}{k}+\frac{c_2}{k^2}+\frac{c_3}{k^3}+\ldots \;\;\;;\;\;\;
%\Pi_k\,=\,C_0+\frac{C_1}{k}+\frac{C_2}{k^2}+\frac{C_3}{k^3}+\ldots
%\]
%
%De una estimaci\'on para $\pi$.
%Compare este resultado con la mejor evaluaci\'on que se puede obtener con el algoritmo %de 
%Arq\'{\i}medes.

% 7
\prob{} Escriba un programa para calcular un valor aproximado de $\pi$
utilizando 
\begin{enumerate}
\item la f\'ormula recurrente de Arqu\'imedes, que  acota $\pi$ entre
$P_n$ y $p_n$, con $p_n < \pi < P_n$. Siendo $p_n$ y $P_n$ 
los per\'imetros de los pol\'igonos regulares de $n$ lados inscriptos y 
circunscriptos, respectivamente, en la circunsferencia de radio $1/2$. 
La f\'ormula de recurrencia que encontr\'o Arqu\'imides es la siguiente:
\begin{align*}
P_{2n} = \frac{2 p_n P_n}{p_n + P_n} \\
p_{2n} = \sqrt{P_{2n} p_n}
\end{align*}
Usando los valores $P_6=2\sqrt{3}$ y $p_6 = 3$, correspondientes al hex\'agono, 
escriba un programa que realice 20 iteraciones, con $n = 6\times 2^k$, y 
$k=1, \cdots, 20$, y escriba los resultados en pantalla.

\item la productoria de Wallis
\[
\frac{\pi}{2} = \prod_{n=1}^\infty \frac{(2n)^2}{(2n)^2-1} =
\frac43\frac{16}{15}\frac{36}{35}\frac{64}{63}\dots 
%\frac21\frac23\frac43\frac45\frac65\frac67\dots 
\]
Calcule el valor de $\pi$ truncando la productoria a $10^6$ factores.

\end{enumerate}


%%%%%%%%%%  comienzo guia_c


\prob{} Modifique el programa que convierte numeros decimales a la
base $b$, con $b \le 16$, de manera que utilice {\tt SELECT CASE}
para asignar los d\'igitos (del resultado en base $b$). 
Utilice {\tt CASE DEFAULT} (para asignar los d\'igitos del 0 al 9).

\prob{} Dado el arreglo {\tt a}, declarado de la siguiente manera  \\ 

{\fontfamily{phv}\selectfont
\hspace*{2cm}   real (kind(1.0)), dimension(50,20) :: a  \\ }

\noindent
escriba (papel y l\'apiz) {\em secciones} de este arreglo representando:
\begin{enumerate}
\item la primera fila de {\tt a};
\item la \'ultima columna de {\tt a};
\item un elemento de por medio en cada fila y columna.
\end{enumerate}


\prob{} Escriba un programa, que utilizando una subrutina, multiplique un
vector de $N$ elementos, por una matriz de $N\times N$. El programa debe
preguntar el valor de $N$ y luego definir los arreglos, y
darle valores iniciales tal que, la matriz sea triangular superior, con todos
sus elementos igual a 1, excepto los de la diagonal que toman valor 3, el vector
tendra todos sus elementos pares igual a 2, y los impares igual a 3. 
No utilice {\tt DO} para las inicializar el vector, ni {\tt DO} anidados para
inicializar la matriz. 


\prob{} Dado el siguiente programa y su correspondiente subrutina:

\vspace{.3cm}

\begin{minipage}{0.5\textwidth}
{\fontfamily{phv}\selectfont
\hspace*{1cm}   program test  \\
\hspace*{1cm}   implicit none  \\
\hspace*{1cm}   real(kind(1.)) :: x, y,s  \\
\hspace*{1cm}   integer        :: i,j  \\
\hspace*{1cm}   i = 6; j = 3; x = 4.; y = 8.   \\
\hspace*{1cm}   call sum(i,x,s)  \\
\hspace*{1cm}   write(*,'(F4.1," + ",F4.1," = ",F5.1)') real(i),x,s   \\
\hspace*{1cm}   end program test}
\end{minipage}\hfill
\begin{minipage}{0.5\textwidth}
{\fontfamily{phv}\selectfont
\hspace*{2cm}   \phantom{a}  \\
\hspace*{2cm}   subroutine sum(z,w,ss)  \\
\hspace*{2cm}   implicit none  \\
\hspace*{2cm}   real(kind(1.)), INTENT(IN) \hspace{.3cm}  :: z, w  \\
\hspace*{2cm}   real(kind(1.)), INTENT(OUT) :: ss  \\
\hspace*{2cm}   ss = z + w  \\
\hspace*{2cm}   end subroutine sum   } \\
\end{minipage}

\vspace{.3cm}
\noindent
Escriba, compile y ejecute el programa de las siguiente manera:
\begin{enumerate}
\item con el programa tal como esta, incluyendo la subrutina en el mismo archivo o en 
archivo separado.

\item modificando el programa con la cl\'ausula {\tt CONTAINS}, para que {\em contenga} a
la subrutina.

\item creando un modulo que contenga la subrutina, y usando este en el programa.

%\item agregando al programa, antes de la primera l\'inea ejecutable, 
%      el siguiente {\tt INTERFACE BLOCK}: 
%
%{\fontfamily{phv}\selectfont
%\hspace*{3cm}  INTERFACE  \\
%\hspace*{3.5cm}  subroutine sum(z,w,ss)  \\
%\hspace*{3.5cm}  implicit none  \\
%\hspace*{3.5cm}  real(kind(1.)), INTENT(IN) \hspace{.3cm}  :: z, w  \\
%\hspace*{3.5cm}  real(kind(1.)), INTENT(OUT) :: ss  \\
%\hspace*{3.5cm}  end subroutine sum(z,w,ss)  \\
%\hspace*{3cm}  END INTERFACE} \\
%\noindent
%      incluyendo la subrutina en el mismo archivo o en archivo separado.

\end{enumerate}
\noindent
Verifique que en el primer caso compila sin errores y produce {\em resultados incorrectos},
y en los otros casos no.



\prob{} Escriba un programa para calcular la posici\'on y la 
velocidad en funci\'on del tiempo, para un problema de tiro oblicuo.
Debe preguntar el \'angulo (en grados) y la velocidad 
inicial (en m/seg.), asumiendo que el proyectil parte del origen. 
Elija el incremento temporal ($\Delta t$) de manera que la gr\'afica
tenga 600 puntos y abarque el intervalo entre el disparo y el instante en
que el proyectil vuelve a tener altura 0. Utilice {\em funciones} para calcular
la posici\'on y la velocidad. Escriba la salida del programa en un archivo
de texto, con 5 columnas ($t$, $x(t)$, $y(t)$,$v_x(t)$,$v_y(t)$). La primera
l\'inea debe comenzar con \#, e incluir la descripci\'on de los datos
de cada columna. Los datos en la tabla deben tener 6 cifras significativas y
estar escritos en notaci\'on exponencial. Grafique $x(t)$, $y(t)$ y $v_y(t)$ 
en funci\'on de $t$, y la trayectoria del proyectil, utilizando {\em gnuplot}.



\bigskip
\bigskip
\newpage
\noindent {\bf Ejercicios Complementarios}

% 6
\prob{} Se pretende calcular las sumas $S_N = \sum_{k=1}^N a_k$, 
donde $N$ es un n\'umero natural. Llamemos $S^{\prime}_N$ al valor calculado
que se logra de hacer {\tt (float)}$(S_{N-1} + a_N)$.
Sea $S_N = \sum_{k=1}^N 1/k$. Mostrar que $S^{\prime}_N$
se estaciona a partir de alg\'un $N$ suficientemente grande. Deducir
que a partir de entonces $S_N \neq S_N^{\prime}$. Hacer un programa
que determine el valor a partir del cual $S_N^{\prime}$ se
estaciona.
% 

\prob{} Para calcular un valor aproximado de $\pi$ utilizaremos la siguiente serie
infinita alternante
\begin{equation}
\sum_{n=0}^{\infty} \frac{(-1)^n}{2n+1} = \frac{\pi}{4} \label{serie}
\end{equation}
Recordemos que una cota superior para el error cometido al truncar una serie
alternante (de valor absoluto decreciente) est\'a dado por el valor absoluto del
primer t\'ermino despreciado.
Escriba un programa que ingrese el n\'umero de cifras decimales exactas con que se
desea el valor de $\pi$ (entre 1 y 5 cifras) y devuelva en pantalla el n\'umero
de t\'errminos que deben incluirse en la serie (\ref{serie}) para obtener dicha
precisi\'on y a rengl\'on siguiente el valor de $\pi$ obtenido de esta forma, truncado 
el resultado al n\'umero de cifras pedido.


% 
%\prob{} Escribir un programa que pida un n\'umero e imprima por pantalla 
%su tabla de sumar.

% 
\prob{} Escribir un programa que pida una contrase\~na de tres d\'igitos y permita
leer tres intentos.  Si el usuario da la contrase\~na correcta responde responde
``Correcto'' y queda inactivo, con este mensaje.  En caso contrario el programa
escribe ``Lo siento, contrase\~na equivocada'' y se cierra de inmediato.

% 
\prob{} Escribir un programa que, dado un a\~no y el nombre de un mes, 
saque por pantalla el n\'umero de d\'ias del mes (tenga en cuenta que 
algunos a\~nos son bisiestos). 

% 
\prob{} Escriba un programa que genere secuencialmente 10 archivos con nombre de salida diferentes (dependiendo del valor que tome alg\'un par\'ametro). En cada archivo, escriba bajo la forma de dupla $(x,f(x))$ una funci\'on evaluada en $N$ puntos y que tambi\'en dependa del par\'ametro (por ejemplo $y=sin(\omega \pi x)$, con $\omega = 1, 2, 3, \ldots $). El \textit{loop} debe cerrar cada archivo luego de escribir en \'el. En el mismo programa o en otro, construya otro \textit{loop} que abra secuencialmente los archivos y que (sin borrar los datos escritos) agregue otros $N$ puntos de la funci\'on $(x,f(x))$.

\end{document}

