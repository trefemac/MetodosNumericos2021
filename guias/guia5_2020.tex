\documentclass[a4paper]{article}
\usepackage[latin1]{inputenc}
%\usepackage{babel}
\topmargin 0truecm
\textheight 22truecm
\textwidth 16truecm
\oddsidemargin 0truecm
\parskip 1truemm
%\parsep 1truemm
\parindent 0em
\usepackage{amssymb}
\usepackage{color}

\def\d{\,{\rm d}}


%   COLORS
\definecolor{light-blue}{rgb}{0.8,0.85,1}
\definecolor{mygrey}{gray}{0.75}
%
%

\def\RR{{\mathbb R}}
\def\CC{{\mathbb C}}
\def\NN{{\mathbb N}}
\def\ZZ{{\mathbb Z}}
\def\EE{{\mathbb E}}
\newcounter{problema}
\newcommand{\prob}[1]{\vspace{0.33cm}\stepcounter{problema}
                 \noindent{\bf Problema \arabic{problema}:}{\it #1}}

\begin{document}
\renewcommand{\labelenumi}{\bf \alph{enumi})}
\renewcommand{\labelenumii}{\alph{enumi}$_\arabic{enumii}$)}
\thispagestyle{empty}
\begin{center}
{\large {\bf M\'etodos Num\'ericos}}\\
{\small Aula Virtual: http://www.famaf.proed.unc.edu.ar/course/view.php?id=451} \\

\rule{0em}{2.em}{\bf Gu\'{i}a 5} \\ Mayo de 2020
\end{center}



\prob{} Sea $F(x) = x e^x$. Eval\'ue $f^\prime (2)$ mediante la f\'ormula centrada de tres puntos

\[
f^\prime(x) = \frac{f(x+h) - f(x-h)}{2h} + \mathcal{O}(h^2)
\]
\noindent
para distintos valores de $h$ y calcule el incremento \'optimo $h_o$
teniendo en cuenta los errores de truncamiento y redondeo. Grafique
el error (usando el valor exacto de la derivada) versus $h$ (elija
$h = 10^{-k}$, con $k$ entero, y grafique usando escala {\em log-log}).



\prob{ F\'ormula de la derivada num\'erica de 5 puntos:} 
Para funciones $f(x)$ que son suficientemente diferenciables en $x=c$

%\marginpar{ \colorbox{light-blue}{entregar}}

\begin{enumerate}
\item muestre que se  puede aproximar
$f'(c)$ con la f\'ormula centrada de 5 puntos como:
\[
f'(c)\,=\,\frac{1}{12 h} \left(f(c-2 h)-8 f(c-h)+8 f(c+h)-f(c+2 h)
\right)\,+\,O(h^4) ,
\]
%\[
%f''(c)\,=\,\frac{1}{12 h^2} \left(-f(c-2 h)+16 f(c-h)-30 f(c)+16 f(c+h)-f(c+2 h)
%\right)\,+\,O(h^4) .
%\]


\item muestre que con esta f\'ormula se obtiene la derivada primera exacta para polinomios de grado $\le 4$.


\end{enumerate}



\prob{} Use las f\'ormulas {\em hacia adelante}, {\em centrado} y {\em de 5 puntos} para
calcular las derivadas de $\cos x$ y $e^x$, en $x=0.1, 1.$ y $100$.  
\begin{enumerate}
\item Escriba en archivo el valor de la derivada y el error relativo, $E$, en funci\'on 
de $h$. Elija valores del {\em paso} $h$ entre 0.1 y $\epsilon_m$.
\item Haga un gr\'afico {\em log--log} de $E$  versus $h$, y verifique si el n\'umero
de cifras decimales que obtiene coincide con las estimaciones hechas en el te\'orico.
\item Identifique las regiones donde domina el error del algoritmo y el error de redondeo,
respectivamente. Las pendientes que se observan, corresponden a las predichas en el 
te\'orico?
\end{enumerate}

\prob{} En el archivo {\em pos.dat} (que se encuentra en el directorio {\em /opt/metnum/}) 
se dispone de un conjunto de datos experimentales de la posici\'on de un m\'ovil
que se desplaza en l\'inea recta. El mismo tiene dos columnas, siendo la primera el tiempo de 
la medici\'on, $t_i$ y la segunda la posici\'on, $x(t_i)$. 
\begin{enumerate}
\item Escriba un programa que lea los datos del archivo y calcule la velocidad del m\'ovil 
      para los mismos tiempos, $t_i$. Utilice la f\'ormula de 3 puntos. 
      Preste especial atenci\'on a los puntos de los bordes del intervalo.

\item Repita el punto {\bf a)}  utilizando las f\'ormulas de 5 puntos.

%\item Modifique el programa anterior de manera que tambi\'en calcule la derivada segunda.

\end{enumerate}




%%%%%%%%%%%%%%%%%%%%%%%%%%%%%%%%%%%%%%
% problemas teoricos


\noindent {\bf Ejercicios Complementarios}


\prob{ Derivada segunda:} Deduzca la f\'ormula centrada equiespaciada de tres puntos
para la derivada segunda $f"(x_0)$. Incluya una cota para el error absoluto.

\prob{ Interpolaci\'on y diferenciaci\'on:}
Se conoce el valor de $f(x)$ en tres puntos $x_0,x_1,x_2$. Escriba el polinomio
interpolante $P_2(x)$ en la forma de Lagrange.
Asuma que aproximamos $f'(x_i)$ por $P_2'(x_i)$,
\begin{enumerate}
\item Muestre que si tomamos
$x_0=c-h,\,x_1=c,\,x_2=c+h$ obtenemos la expresi\'on del algoritmo centrado de tres
puntos para $f'(c)$.
\item Muestre que, en general, esta proximaci\'on arroja el algoritmo de tres puntos.
Re-obtenga la f\'ormula dada en el te\'orico para $x_0=c-h_1,\,x_1=c,\,x_2=c+h_2$.
Obtenga una expresi\'on para las derivadas en extremos del intervalo $[a,b]$,
$f'(a)$ con $x_0=a,\,x_1=a+h,\,x_2=a+2 h$ y $f'(b)$ con $x_0=b,\,x_1=b-h,\,x_2=b-2h$.
\item Generalice a 5 puntos y re-obtenga el algoritmo centrado y equiespaciado en
este caso.
\end{enumerate}


\end{document}
