\documentclass[a4paper]{article}
\usepackage[latin1]{inputenc}
%\usepackage[spanish]{babel}
\usepackage{amsmath}

\topmargin 0truecm
\textheight 22truecm
\textwidth 16truecm
\oddsidemargin 0truecm
\parskip 1truemm
%\parsep 1truemm
\parindent 0em
\usepackage{amssymb}

\def\d{\,{\rm d}}

\def\RR{{\mathbb R}}
\def\CC{{\mathbb C}}
\def\NN{{\mathbb N}}
\def\ZZ{{\mathbb Z}}
\def\EE{{\mathbb E}}
\newcounter{problema}
\newcommand{\prob}[1]{\vspace{0.33cm}\stepcounter{problema}
                 \noindent{\bf Problema \arabic{problema}:}{\it #1}}

\newcommand{\itemEq}[1]{%
        \begingroup%
        \setlength{\abovedisplayskip}{0pt}%
        \setlength{\belowdisplayskip}{0pt}%
        \parbox[c]{\linewidth}{\begin{flalign*}#1&&\end{flalign*}}%
        \endgroup}


\begin{document}
\renewcommand{\labelenumi}{\bf \alph{enumi})}
\renewcommand{\labelenumii}{\alph{enumi}$_\arabic{enumii}$)}

\thispagestyle{empty}
\begin{center}
{\large {\bf M\'etodos Num\'ericos}}\\
{\small Aula Virtual: http://www.famaf.proed.unc.edu.ar/course/view.php?id=452} \\

\rule{0em}{2.em{\bf}Gu\'ia 7} \\  Junio de 2020
\end{center}



\prob{}
Escriba un programa en Fortran que le permita resolver num�ricamente
el problema de valores iniciales de la forma,
\[
\frac{dy}{dt} = f(t,y),\;\;\;\;\;\; a\leq t \leq b,\;\;\;\;\; y(a) = \alpha
\]
utilizando los m�todos de Euler, Runge Kutta de 2$^{\circ}$ orden y Runge 
Kutta de 4$^{\circ}$ orden en el intervalo $a\le t\le b$ con un paso de
integraci�n $h$. 
El programa debe usar un m\'odulo de precisi\'on y  
 un m\'odulo que contenga tres subrutinas, una para el paso de integraci\'on de cada m\'etodo,
y la funci\'on f(t,y). 
Utilice una variable entera (entrada por teclado) para seleccionar el m\'etodo 
a utilizar, y de acuerdo a \'este, el programa debe generar un archivo de salida 
(con nombre indicativo del m\'etodo) con dos columnas separadas por 3 espacios, en formato exponencial, 
con 7 cifras significativas: $t_i= a + i\,h$ y la correspondiente aproximaci\'on $w_i$ a la soluci\'on 
exacta $y(t_i)$.


\prob{}
Utilizando el programa realizado en el problema 1, resuelva
con los tres m\'etodos dados en el te\'orico (Euler, Runge Kutta de 
segunda orden y Runge Kutta de cuarta orden) el siguiente problema de 
valores iniciales:
\[
\frac{dy}{dt} = -y+\sin(2\pi t), \;\;\;\;\;\; 0 \le t \le 1\; , 
\;\;\;\;\; y(0) = 1.0
\]
en el intervalo $0 \le t \le 1$ con un paso de integraci�n $h=0.1$. 
Sabiendo que la soluci�n exacta es:
\[
x_e(t)=\Bigl(1+\frac{2\pi}{1+4\pi^2}\Bigr)e^{-t}+\frac{\sin(2\pi t)-2\pi
    \cos(2\pi t)}{1+4\pi^2},
\]
modifique el programa de forma tal que calcule, para cada m\'etodo,
tambi�n el error global $\epsilon(t) = |y(t)-y_e(t)|$. Usando gnuplot observe que su soluci�n se aproxima a la soluci�n exacta. 
Grafique $\epsilon(t)$ usando $h=0.01$ y $h=0.005$ para cada m\'etodo (no olvide hacer gr�ficas completas, en color, con leyendas apropiadas, t�tulo, ejes y rangos adecuados  que muestren claramente la conclusi�n del problema).
Discuta los resultados.

\smallskip

\prob{}
Considere el problema de valor inicial:
\[
\frac{dy}{dx} = \sin{(y)},\;\;\;\;\;\; 0\le t\le 20.0, \;\;\;\;\; y(0)=\alpha
\]
Resu\'elvalo para los siguientes valores iniciales $\alpha_1=0.5$,
$\alpha_2=2.0$, $\alpha_3= \pi$, $\alpha_4=3.6$ $\alpha_5=5.5$ y 
$\alpha_6=2\pi$, en todos los casos con $h=0.1$. 
Para cada valor inicial genere un archivo de salida
como el indicado en el problema 1 (solo para RK4). Luego grafique
simult�neamente las seis curvas aproximadas a las soluciones de los
seis problemas de valores iniciales (no olvide hacer un gr�fico de calidad, completo). Analice.


\prob{ M\'etodo de Runge-Kutta de orden 4: } Muestre que la elecci\'on dada en el
te\'orico para los pesos $\vec{b}$, los nodos $\vec{c}$ y la matriz {\bf A} para el m\'etodo RK4:
\[
\vec{b}=(1/6,1/3,1/3,1/6)  \;\;\;;\;\;\; \vec{c}=(0,1/2,1/2,1)
\;\;\;;\;\;\; a_{2,1}\,=\,a_{3,2}\,=\,1/2 \;;\;a_{4,3}\,=\,1
\]

\noindent conduce a las ecuaciones RK4 ``cl\'asicas" dadas en clase.

\smallskip

\prob{}
Considere el problema de valores iniciales para la ecuaci�n de la din�mica
de un p�ndulo simple de longitud $l$
\[
\frac{d^2\theta}{d t^2} = - \frac{g}{l} \sin{(\theta)}, \quad
\theta(0)=\theta_0, \quad \frac{d\theta}{d t}(0)= \dot{\theta}_0,
\]
donde $g$ es la acelaraci�n de la gravedad. Definiendo  $u= \dot{\theta}$ esta ecuaci�n
de segundo orden se puede escribir como un sistema de dos ecuaciones de primer
orden
\begin{align}
\frac{d\theta}{d t} &= u \label{uno}\\
\frac{d u}{d t} &= - \frac{g}{l} \sin{(\theta)} \label{dos}
\end{align}
mientras que las condiciones iniciales transformadas quedan
$(u(0),\theta(0))=(\dot{\theta}_0,\theta_0).$

Utilice y modifique el programa del problema 1 de forma tal que resuelva 
ahora este sistema de dos ecuaciones diferenciales ordinarias acopladas 
utilizando el m\'etodo de Runge Kutta de 4$^{\circ}$ orden con $g=10 m/s^2$ 
y $l=1 m.$ Ahora la salida debe ser un archivo de tres columnas $t$, 
$\theta(t)$ y $u(t)$.
\begin{enumerate}
\item Grafique $\theta$ vs. $t,$ para $0\le t\le 10,$ con las siguientes
condiciones iniciales: a) $u(0)=0$ y $\theta(0)=0.5$ y b) $u(0)=0$ y
$\theta(0)=0.25$

\item Modifique el programa para que calcule la energ\'ia del sistema en cada paso,
y la escriba en un archivo de salida. Para las condiciones del inciso anterior
grafique la energ\'ia vs. $t$. Analice la conservaci\'on para distintos valores de $h$.


\item Para las condiciones iniciales $\theta(0)=\theta_0,$ y $u(0)=0,$ y s�lo
cuando $\theta_0\ll 1,$ las ecuaciones de movimiento del p\'endulo se pueden
aproximar por las siguientes:
\begin{align}
\frac{d\theta}{d t} &= u\\
\frac{d u}{d t} &= - \frac{g}{l} \theta .
\end{align}
\noindent
Modifique el programa para resolver estas ecuaciones y compare con la
 soluci�n exacta ($\theta(t) = \theta_0 \cos(\sqrt{10}t).$)
 Para verificar esto graficar la diferencia $\theta(t)-\theta_0
\cos(\sqrt{10}t),$ para $0\le t\le 10,$ en los casos $\theta_0=1$ y
$\theta_0=10^{-2}$. En los mismos gr\'aficos comparar con la soluci\'on
exacta del problema, i.e. con la soluci\'on num\'erica de las ecuaciones (\ref{uno}) y
(\ref{dos}).

\end{enumerate}

\smallskip


\begin{center}
{\bf Problemas complementarios}
\end{center}

\smallskip

\prob{  }  Use el m\'etodo del disparo para resolver los siguientes problemas de 
frontera con una tolerancia de $10^{-5}$. Se da un valor tentativo inicial de $h$ 
y la soluci\'on exacta  para comparaci\'on.
 
\begin{enumerate}
\item $1\leq t\leq 2$, comience con $h=0.5$
\[
\ddot{x}\,=\,-(\dot{x})^2 \,,\;\;\;x(1)=0\;,\;\;x(2)=\ln{(2)} \,.
\]
\noindent Soluci\'on exacta $x=\ln{(t})$.

\item $-1\leq t\leq 0$, comience con $h=0.25$
\[
\ddot{x}\,=\,2 x^3\,,\;\;\;x(-1)=\frac{1}{2}\;,\;\;x(0)=\frac{1}{3} \,.
\]
\noindent Soluci\'on exacta $x=1/(t+3)$.

\item $1\leq t\leq 2$, comience con $h=0.05$
\[
\ddot{x}\,=\,\frac{(t\,\dot{x} )^2\,-9 x^2+4 t^6}{t^5},\;\;\;x(1)=0\;,\;\;x(2)=\ln{(256)} \,.
\]
\noindent Soluci\'on exacta $x=t^3\,\ln{(t})$.



\end{enumerate}



\prob{}  Considere la siguiente ecuaci\'on diferencial
\[
y^{\prime \prime} = \frac{1}{8} \left( 32 + 2 x^3 - y y^\prime \right)    \qquad \qquad \mbox{para }
1 \le x \le 3
\]
\noindent
\begin{enumerate}
\item Utilice el m�todo de Runge-Kutta de 4$^{\circ}$ orden en el intervalo $1 \le
x\le 3$ para resolver esta ecuaci\'on con las condiciones iniciales $y(1) = 17$, $y^\prime (1) = 0$.
Encuentre, adem\'as $y^\prime (3)$.

\item Repita el inciso anterior, pero con las condiciones iniciales $y(1) = 17$, $y^\prime (1) = -40$.

\item Use ahora el m\'etodo de disparo para resolver la misma ecuaci\'on diferencial con las condiciones 
de borde $y(1) = 17$, $y^\prime (3) = 0$. Con la informaci\'on de los incisos anteriores implemente
un m\'etodo de bisecci\'on con una tolerancia de $10^{-10}$. Escriba en archivo el n\'umero de la 
iteraci\'on y el valor de la derivada en $x=3$, y una vez encontrada la soluci\'on, en 
otro archivo, escriba $x$, e $y(x)$, para una grilla de 400 valores equiespaciados de $x$, entre
0 y 3. Grafique la convergencia y la soluci\'on.


\end{enumerate} 


\prob{}
La llamada {\em ecuaci�n log\'{\i}stica}
$$
\frac{dN}{dt}= r\,N \left(1-\frac{N}{K}\right)
$$
describe el crecimiento autolimitado de una poblaci�n dada (suponiendo que no
interact�a con otras especies y que tiene fuentes limitadas de alimentos). Fue
propuesta por Verhulst en 1838 y permite describir al menos cualitativamente
varios fen�menos poblacionales observados en la naturaleza. En esta ecuaci�n
$N(t)$ es el n�mero de individuos de la colonia al tiempo $t$ y $K$ es una
constante positiva.

Una soluci�n $N^*$ se dice estacionaria si se satisface que $dN^*/dt=0$, y por
ende no cambia en el tiempo. Para esta ecuaci�n es f�cil verificar que
s�lo existen dos soluciones estacionarias: $N_1^*=0$ y $N_2^*=K$.

Determine cu�l de las dos soluciones estacionarias es estable y cu�l inestable
resolviendo num�ricamente la ecuaci�n diferencial con el m�todo
Runge-Kutta de cuarto orden para $r=2$, $K=100$, en el intervalo $0\le t \le 50$
con $h=0.1$ y considerando cinco condiciones iniciales diferentes: a) $N(0)= 0$,
b) $N(0)=2$, c) $N(0)=50$, d) $N(0)= 120$ y d) $N(0)=200$.  Grafique
simult�neamente las cinco soluciones $t$ vs.  $N(t)$ en el intevalo $0\le t\le
50$ en un gr�fico completo.

\end{document}
