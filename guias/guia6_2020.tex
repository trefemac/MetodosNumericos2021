\documentclass[a4paper]{article}
\usepackage[latin1]{inputenc}
%\usepackage[spanish]{babel}
\usepackage{amsmath}

\topmargin 0truecm
\textheight 22truecm
\textwidth 17truecm
\oddsidemargin 0truecm
\parskip 1truemm
%\parsep 1truemm
\parindent 0em
\usepackage{amssymb}

\def\d{\,{\rm d}}
\def\RR{{\mathbb R}}
\def\CC{{\mathbb C}}
\def\NN{{\mathbb N}}
\def\ZZ{{\mathbb Z}}
\def\EE{{\mathbb E}}
\newcounter{problema}
\newcommand{\prob}[1]{\vspace{0.33cm}\stepcounter{problema}
                 \noindent{\bf Problema \arabic{problema}:}{\it #1}}

\newcommand{\itemEq}[1]{%
        \begingroup%
        \setlength{\abovedisplayskip}{0pt}%
        \setlength{\belowdisplayskip}{0pt}%
        \parbox[c]{\linewidth}{\begin{flalign*}#1&&\end{flalign*}}%
        \endgroup}


\begin{document}
\renewcommand{\labelenumi}{\bf \alph{enumi})}
\renewcommand{\labelenumii}{\alph{enumi}$_\arabic{enumii}$)}

\thispagestyle{empty}
\begin{center}
{\large {\bf M\'etodos Num\'ericos}}\\

\rule{0em}{2.em{\bf}Gu\'ia 6} \\  Mayo de 2020
\end{center}


\prob{}
Encuentre, haciendo los c\'alculos a mano, las aproximaciones 
a las siguientes integrales definidas utilizando las f\'ormulas del 
trapecio, Simpson y punto medio (simples):
\begin{enumerate}

\item \itemEq{I_1 = \int_0^1 x^4 dx}

\item \itemEq{I_2 = \int_0^{\pi} \sin{(x)} dx} 

\end{enumerate}
Trabaje con siete cifras significativas. Calcule el error absoluto y el error 
relativo en cada caso y para cada m�todo.

\prob{}  Repita el Problema 1 dividiendo el intervalo de integraci�n en 
dos subintervalos de igual tama�o:
\begin{enumerate}

\item \itemEq{I_1=\int_0^1 x^4 dx+ \int_1^2 x^4 dx}

\item \itemEq{I_2=\int_0^{\pi/2}\sin{(x)}dx+\int_{\pi/2}^{\pi}\sin{(x)} dx} 

\end{enumerate}
Trabaje con siete cifras significativas. Calcule el error absoluto y 
el error relativo en cada caso y compare con los resultados del problema 1.


\prob{}
\begin{enumerate}
\item Realice un programa que encuentre  las aproximaciones num�ricas $S_T$ y $S_S$ a la integral:
\[
I = \int_a ^b f(x) dx
\]
utilizando  la regla compuesta del {\em trapecio} y la de {\em Simpson}, respectivamente.  
Debe evaluar el integrando $f(x)$ en $n+1$ puntos equiespaciados  $x_i$ ($i=0,1,2, \ldots , n$) 
con espaciamiento $h=(b-a)/n$. La salida debe ser las dos aproximaciones.
El programa
debe escribirse en  doble precisi�n y  debe utilizar un m\'odulo de precisi\'on y un m\'odulo con dos
subrutinas \verb!trapecio! y \verb!simpson!. 

\item Utilizando el programa realizado en el punto  anterior, encuentre las
dos aproximaciones num\'ericas $S_T$ y $S_S$ a la integral:
\[
I = \int _0 ^1 e^{-x} dx \, .
\]
Utilice un espaciamiento $h_1=0,05$ em ambos casos. Luego repita el procedimiento disminuyendo su 
espaciamiento a la mitad, $h_2=h_1/2=0,025$.


\item Teniendo en cuenta que es posible conocer el resultado
exacto de la integral en cuesti\'on, eval\'ue el error
$\varepsilon(h)=|S-I|$, para $h=0.05$ y $h=0.025$. Verifique que
el cociente de precisi�n, definido como 
\[
Q = \frac{\varepsilon(h)}{\varepsilon(h/2)} \, ,
\]
toma un valor aproximado a $4$ cuando se usa la regla compuesta del 
{\em trapecio}, y un valor aproximado a $16$ cuando se usa la regla 
compuesta de {\em Simpson}. Teniendo en cuenta la expresi\'on del 
error de truncamiento en cada caso, justifique este resultado.
\end{enumerate}



 {\em Consejo:} tener cuidado con {\em Simpson} en elegir siempre un n�mero 
par de intervalos, {\it i.e}, un n�mero impar de puntos (pruebe con una integral 
que conoce su resultado qu� resultado le da cuando usa un n�mero impar de intervalos).


\prob{} Usando el programa del punto a) del problema 3, aproxime las 
si\-guien\-tes integrales con un error relativo menor a $10^{-7}$ 
(utilice doble precisi\'on). Indique, en cada caso, el n\'umero de 
puntos utilizado.\\
\begin{enumerate}

\item \itemEq{ \int_0^{0.5} \frac{2}{x-4} \; dx } 

\item \itemEq{  \int_1^{1.5} x^2 \log x \; dx } 

\end{enumerate}

{\it Sugerencia 1:} antes de programar grafique el integrando y analice su comportamiento 
dentro del intervalo dado. Si puede estime el resultado, al menos el signo del \'area 
esperada. Este an\'alisis previo  da criterio para saber si el resultado num\'erico es correcto.\\

{\it Sugerencia 2:} abra archivos de escritura con  atribuciones  como status, etc... y escriba 
con formato de escritura exponencial, en columnas separadas con los espacios que elija. Es 
buena costumbre tambi\'en escribir el nombre arriba de cada columna de datos.


\prob{ Cuadratura de Gauss--Legendre. Comparaci\'on con otros m\'etodos}

\begin{enumerate}
\item Escriba un programa que calcule la integral
\begin{equation*}
I = \int_0^1 e^{-t} dt = 1 - e^{-1}  \nonumber              
\end{equation*}
utilizando la cuadratura de Gauss--Legendre. Para calcular los puntos y pesos necesarios
use la subrutina {\em gauss} que se provee.

\item Calcule el error relativo $\epsilon = |(num\acute{e}rico - 
exacto)/exacto |$ para distintos valores del n\'umero de puntos de integraci\'on, $N$. 
Considere valores de $N$ consecutivos hasta 100, y luego de 100 en 100 hasta $N=3000$. 
Repita el c\'alculo empleando los algoritmos de Simpson y Trapecio utilizando el programa
del problema 3.

\item Haga un gr\'afico {\em log-log} comparativo de  los errores relativos {\em versus} $N$. Observe 
que

\[ 
\epsilon \approx CN^\alpha => \log \epsilon = \alpha \log N + constante.
\]
\noindent
Esto significa que una dependencia como ley de potencia aparece como
una l\'inea recta en un gr\'afico {\em log-log}.

\item Use el gr\'afico para estimar las leyes de potencia de la dependencia del error
$\epsilon$ con el n\'umero de puntos $N$ y para determinar el n\'umero cifras 
decimales de precisi\'on en cada m\'etodo. Haga esto tanto para el error del algoritmo como para el
de redondeo.

\item Repita los incisos anteriores usando doble precisi\'on.

\end{enumerate}

\prob{}  Modifique el programa del problema 3.a) para leer  archivos de datos que contienen 
en la primer columna los puntos de integraci�n y en la segunda columna el valor de la funci�n 
a integrar. Copie  de  \verb!/opt/metnum! a  su directorio los archivos \verb!mediciones1-p4.dat!  
y \verb!mediciones2-p4.dat! y realice  ambas integrales. El primer archivo corresponde a un 
muestreo con 629 puntos y el segundo con 10001 puntos 
del mismo intervalo de integraci�n. Observe los datos medidos y su periodicidad.  \\[1cm]



%\newpage

\noindent {\bf Ejercicios Complementarios}


\prob{}  Idem problema 4, para las siguientes integrales: 
\begin{enumerate}
\item \itemEq{ \int_{0.5}^1 x^4 dx } 

\item \itemEq{ \int_0^{\frac{\pi}{4}} x \sin x dx }
  
\end{enumerate}



\prob{ Cuadratura de Gauss-Legendre de 3 puntos}
\begin{enumerate}
\item Calcule los puntos y pesos correspondientes a la cuadratura de Gauss-Legendre 
(G-L)
de 3 puntos, esto es, intervalo $[-1,1]$ y funci\'on de peso $w(x)=1$.
\item Utilice cuadraturas de  G-L de 2 y 3 puntos  para estimar  la 
integral
\[
\int_1^{3/2}  \, e^{-x^2}\,dx\,,
\]

\noindent y compare los errores relativos de ambas.
\item Realice ahora la integral utilizando los c\'odigos que escribi\'o para 
cuadraturas de trapezoide y Simpson al mismo error relativo que G-L de 3 puntos. 
Qu� valores de $N$ debe usar en cada caso?
\end{enumerate}

\prob{ Integraci\'on num\'erica en dos dimensiones } 
\begin{enumerate}
\item Haga un programa que integre funciones en la region $a\le x \le b\,;\;
 c \le y \le d$ siguiendo el c\'odigo delineado en la clase te\'orica, usando el
 m\'etodo de cuadratura de Simpson en cada coordenada.
 \item Evalue n\'umericamente con no menos de 8  cifras significativas las integrales
 \[
 \int_0^2\,dx\, \int_0^1\,dy\, e^{-x\,y} \;\;\;\;\;\;\;;\;\;\;\;\;\;\;
 \int_{1.4}^2\,dx\, \int_1^{3/2}\,dy\, \ln{(x+2 y)}
 \]
 \item Modifique el programa para permitir que los l\'{\i}mites de integraci\'on en $y$ sean funci\'on de $x$ y evalue la integral
 \[
 \int_0^1\,dx\, \int_0^{\sqrt{1-x^2}}\,dy\, e^{-x\,y} 
 \]
\end{enumerate}



\prob{ F\'ormula de Integraci\'on de Newton--Cotes}.

Estas f\'ormulas se basan 
en la idea de integrar una funci\'on polinomial en vez de $f(x)$, i.e.
\[
I = \int_a^b f(x) dx \approx \int_a^b P_n(x) dx
\]
\noindent
donde $P_n(x)$ es el polinomio interpolante de grado $n$ para ciertos 
valores de  $x$ que se escogen apro\-pia\-damente en el intevalo $[a,b]$.
Muestre que si se eligen puntos equiespaciados y que incluyan los extremos 
(f\'ormula cerrada), los casos  $n=1$ y $n=2$ corresponden a la regla del
{\em trapecio} y de {\em Simpson}, respectivamente. {\em Ayuda}: use la forma de Lagrange para el
polinomio interpolante, y calcule los pesos de la f\'ormula de integraci\'on 
resultante.

% 
% \prob{ F\'ormula de integraci\'on de Euler-McClaurin} 
% \begin{enumerate}
% 
% \item Deduzca la f\'ormula  de integraci\'on de Euler-McClaurin:
% 
% \[
% \int_a^b \,f(x) \,dx\,=\,h\,\left( \frac{f(a)}{2}+f(x_1)+\ldots+f(x_{N-1})+\frac{f(b)}{2} \right)\,+\,
% \frac{h^2}{12} \,\left( f'(a)-f'(b)\right)\,+\,\mbox{O}(h^4)
% \]
% 
% \item  Muestre que integra exactamente polinomios de grado 3.
% \end{enumerate}
% 
% 
% \prob{ Cuadratura de Gauss-Laguerre de 2 puntos}
% \begin{enumerate}
% \item Calcule los puntos y pesos correspondientes a la cuadratura de Gauss-Laguerre
% (G-La) de 2 puntos, esto es, intervalo $[0,\infty)$ y funci\'on de peso $w(x)=exp(-x)$.
% \item La integral
% \[
% \int_{0}^\infty \, \frac{x^3}{e^{x}-1}\,dx
% \]
% 
% \noindent aparece en el tratamiento de la radiaci\'on de cuerpo negro de Planck que da
% nacimiento a la mec\'anica cu\'antica en el a\~no 1900. Evalue la misma usando
% la cuadratura de Gauss-Laguerre de 2 puntos y calcule el error relativo.
% \end{enumerate}
% 
% \prob{ Cuadratura de Gauss-Hermite de 3 puntos}
% \begin{enumerate}
% \item Calcule los puntos y pesos correspondientes a la cuadratura de Gauss-Hermite (G-H)
% de 3 puntos, esto es, intervalo $(-\infty,\infty)$ y funci\'on de peso 
% $w(x)=exp(-x^2)/\sqrt{\pi}$.
% \item Utilice cuadraturas de  G-H de 2 y 3 puntos  para estimar  la
% integral
% \[
% \int_{-\infty}^\infty \, \frac{e^{-x^2}}{\sqrt{\pi}}\,e^x\,dx\,,
% \]
% 
% \noindent y compare los errores relativos de ambas.
%\end{enumerate}
\end{document}
