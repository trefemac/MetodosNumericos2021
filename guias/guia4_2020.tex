\documentclass[a4paper]{article}
\usepackage[latin1]{inputenc}
%\usepackage{babel}
\topmargin 0truecm
\textheight 22truecm
\textwidth 16truecm
\oddsidemargin 0truecm
\parskip 1truemm
%\parsep 1truemm
\parindent 0em
\usepackage{amssymb}

\def\d{\,{\rm d}}

\def\RR{{\mathbb R}}
\def\CC{{\mathbb C}}
\def\NN{{\mathbb N}}
\def\ZZ{{\mathbb Z}}
\def\EE{{\mathbb E}}
\newcounter{problema}
\newcommand{\prob}[1]{\vspace{0.33cm}\stepcounter{problema}
                 \noindent{\bf Problema \arabic{problema}:}{\it #1}}

\begin{document}
\renewcommand{\labelenumi}{\bf \alph{enumi})}
\renewcommand{\labelenumii}{\alph{enumi}$_\arabic{enumii}$)}
\thispagestyle{empty}
\begin{center}
{\large {\bf M\'etodos Num\'ericos}}\\

\rule{0em}{2.em}{\bf Gu\'{i}a 4} \\ 28 de abril de 2020
\end{center}


% 1
\prob{} Para las siguientes funciones $f(x)$, y siendo $x_0 = 0$, $x_1=0.6$ y $x_2=0.9$, construya anal\'iticamente los polinomios interpolantes de Lagrange de grado 1 y 2 que aproximan la funci\'on en $x=0.45$, y encuentre el error absoluto y relativo correspondiente.
\begin{enumerate}
\item $f(x) = \ln (x+1)$
\item $f(x) = \sqrt{x+1}$
\end{enumerate}

\noindent
Grafique en un archivo postscript ambas funciones, sus polinomios interpolantes y 
la aproximaci\'on de Taylor de grado 2 (entorno a $x_o$) en el rango dado.

% 2 subrutina pl Lagrange
\prob{} Escriba una subrutina que eval\'ue el polinomio interpolante de Lagrange de orden $n$ en un punto $x$, con $x_0 < x < x_n$
, siendo $(x_i, f(x_i))$ los puntos a interpolar. La subrutina debe tener el orden del polinomio, $n$, 
un arreglo de tama\~no $n+1$ con los valores de $x_i$, un arreglo de tama\~no $n+1$ con los valores de $f(x_i)$ y
el valor de $x$ donde se evaluar\'a el polinomio, como argumentos de entrada. Finalmente, el valor del polinomio en
$x$, $P(x)$, ser\'a el \'unico argumento de salida.

Escriba un programa que utilice esta subrutina para evaluar los polinomios interpolantes de Lagrange de las funciones
del problema 1, en $N=200$ puntos equidistantes en el intervalo $[x_0,x_n]$, escribiendo los valores en un archivo. 
Grafique el polinomio y los pares $(x_i,f(x_i))$ (utilizados para interpolar), verificando el resultado. Compare adem\'as
con la expresi\'on anal\'itica (como funci\'on en {\em gnuplot}) de los polinomios.


% 3
\prob{} Se desea aproximar $\cos(x)$ en el intervalo $[0,1]$ con un error
absoluto menor a $1\times 10^{-7}$ para todo $x \in [0,1]$. Usando el teorema
del error de la interpolaci\'on polinomial, encuentre la cantidad m\'inima de
puntos de interpolaci\'on.  Verifique graficando (con {\em gnuplot}, guardando
la gr\'afica en archivo {\em postscript}) el error absoluto en el intervalo
para tres casos particulares de $\{x_i\}$.

% 4
\prob{} Construya anal\'iticamente el polinomio interpolante de Newton para las siguientes funciones. 
De una cota del error absoluto en el intervalo $[x_0,x_n]$.
\begin{enumerate}
\item $f(x) = \exp (2x) \cos(3x)$,\hspace{0.7cm} $x_0=0$, $x_1=0.3$, $x_2=0.6$, $n=2$.
\item $f(x) = \ln(x)$, \hspace{0.7cm}  $x_0=1$, $x_1=1.1$, $x_2=1.3$, $x_3=1.4$, $n=3$.
\end{enumerate}

% 4 subrutina pl Newton
\prob{} Escriba una subrutina que eval\'ue el polinomio interpolante de Newton de orden $n$ en un punto $x$, con $x_0 < x < x_n$
, siendo $(x_i, f(x_i))$ los puntos a interpolar. La subrutina debe tener el orden del polinomio, $n$, 
un arreglo de tama\~no $n+1$ con los valores de $x_i$, un arreglo de tama\~no $n+1$ con los valores de $f(x_i)$ y
el valor de $x$ donde se evaluar\'a el polinomio, como argumentos de entrada. Finalmente, el valor del polinomio en
$x$, $P(x)$, ser\'a el \'unico argumento de salida.

Escriba un programa que utilice esta subrutina para evaluar los polinomios interpolantes de Newton de las funciones
del problema 4, en $N=200$ puntos equidistantes en el intervalo $[x_0,x_n]$, escribiendo los valores en un archivo. 
Grafique el polinomio y los pares $(x_i,f(x_i))$ (utilizados para interpolar), verificando el resultado. Compare adem\'as
con la expresi\'on anal\'itica (como funci\'on en {\em gnuplot}) de los polinomios.



% 4
\prob{} 
Suponga que se conoce la funci�n $f(x) = \frac{1}{1 + 25 x^2}$ en los puntos
$x_i = -1 + (i-1) 2 / n$ para $i=1,n+1$; que est�n dados en el intervalo $[-1,1]$.
Calcule la interpolaci�n por el m�todo de Lagrange para los valores de $n=10, 20, 40$.
Haga un gr�fico en pantalla y en archivo, postscript o png en los rangos
$x=[-1:1]$ e $y=[-1,5: 1,5]$, conteniendo la funci�n original y 
las tres interpolaciones evaluadas en 200 puntos equidistantes.
Calcule el error m�ximo para cada caso e incluya estos
datos de errores m�ximos en el gr�fico.

{\bf Nota:} En este problema se observa el llamado fen�meno de Runge, en el que la interpolaci�n
por polinomios usando puntos equiespaciados da resultados divergentes.

�Por qu� no hay contradicci�n con el teorema de aproximaci�n de Weierstrass?


{\bf Hint:}
{\it   
	En el programa fortran puede grabar los errores m�ximos en un archivo, digamos:
	datos/maximum-err-values.lod
	
	En particular notar que el comando: maxval(abs(err10)) da el valor m�ximo del valor absoluto
	del arreglo err10.
	
	Luego puede preparar un archivo para hacer los gr�ficos con gnuplot, digamos:
	int\_lag\_runge\_10\_20\_40.lod
	
	Este archivo luego puede ser le\'ido por gnuplot con el comando:
	
	gnuplot int\_lag\_runge\_10\_20\_40.lod 
	
	Dentro del archivo int\_lag\_runge\_10\_20\_40.lod se puede leer los datos grabados
	por el programa fortran con el comando:
	
	load "datos/maximum-err-values.lod"
	
	Estos valores pueden ser incluidos en el gr�fico con comandos como:
	
	set label gprintf("err10 = \%f",err10) at -0.20,-1.05
	
	set label gprintf("err20 = \%f",err20) at -0.20,-1.2
	
	set label gprintf("err40 = \%f",err40) at -0.20,-1.35
	
	
	
	}


\bigskip

\bigskip

%\newpage
\noindent {\bf Ejercicios Complementarios}



%   Evaluacion de polinomios y su derivada: Metodo de Horner 
%
\prob{} Dado el siguiente polinomio
\[
p(x) = -10 + 5 x - 12 x^2  + 6 x^3  - 2 x^4  + x^5 \ ,\]
\noindent
grafique el mismo utilizando {\em gnuplot} y observe que posee 
una \'unica ra\'iz real positiva, encuentre la misma utilizando El m\'etodo de Newton-Raphson. 
%Elija el valor inicial utilizando los teoremas que acotan la regi\'on del espacio complejo donde se encuentran las ra\'ices.
Eval\'ue el polinomio y su derivada en una subrutina utilizando el algoritmo de Horner.


\prob{ Error  de la interpolaci\'on polinomial para puntos equiespaciados:} Usando el teorema dado en el te\'orico, demuestre el siguiente

{\bf corolario}: Sea $f(x) \varepsilon C_{[a,b]}^{(n+1)}$  tal que su derivada $n+1$ es acotada en $[a,b]$: $\exists M>0 / |f^{(n+1)}(x) |< M \;\forall \,x\varepsilon [a,b]$. Definimos
$x_i=a + i \;; i=0,\cdots,n$ donde $ h=(b-a)/n$. Sea $P_n(x)$ es el polinomio interpolante a $f(x)$: $P_n(x_i)=f(x_i)\;,i=0,\cdots ,n$, entonces $\forall\;x\varepsilon [a,b]$ se tiene

\[
\left| f(x) - P_n(x)\right | \leq \frac{M}{4 (n+1)}\;\left(\frac{b-a}{n}\right)^{n+1}
\]

\end{document}
