\documentclass[a4paper]{article}
\usepackage[latin1]{inputenc}
%\usepackage{babel}
\topmargin 0truecm
\textheight 22truecm
\textwidth 16truecm
\oddsidemargin 0truecm
\parskip 1truemm
%\parsep 1truemm
\parindent 0em
\usepackage{amssymb}

\def\d{\,{\rm d}}

\def\RR{{\mathbb R}}
\def\CC{{\mathbb C}}
\def\NN{{\mathbb N}}
\def\ZZ{{\mathbb Z}}
\def\EE{{\mathbb E}}
\newcounter{problema}
\newcommand{\prob}[1]{\vspace{0.33cm}\stepcounter{problema}
                 \noindent{\bf Problema \arabic{problema}:}{\it #1}}

\begin{document}
\renewcommand{\labelenumi}{\bf \alph{enumi})}
\renewcommand{\labelenumii}{\alph{enumi}$_\arabic{enumii}$)}
\thispagestyle{empty}
\begin{center}
{\large {\bf M\'etodos Num\'ericos}}\\
{\small Aula Virtual: http://www.famaf.proed.unc.edu.ar/course/view.php?id=451} \\

\rule{0em}{2.em}{\bf Gu\'{i}a 3} \\ Abril de 2020
\end{center}

% p1

\prob{} Desarrolle un programa para encontrar la ra�z de una funci�n
$f$ utilizando el m�todo de la bisecci�n, dando como datos de entrada
el intervalo inicial $[a,b]$ y la tolerancia $\varepsilon$. $f$ 
debe definirse como una funci�n dentro del programa (usando CONTAINS), o
en un m\'odulo separado.
La salida debe ser 
\begin{itemize}
\item archivo con cinco columnas: $N$, $x_N$, $f(x_N)$, $|b-a|/2$ (error absoluto), y $|b-a|/|a+b|$ (error relativo), 
con 12 cifras significativas para los reales.
\item la aproximaci�n final $x_N$  (en pantalla)
\item el valor final de $f(x_N)$.  (en pantalla)
\item el n\'umero de iteraciones realizadas  (en pantalla)
\end{itemize}

\noindent
Utilice el programa  para

\noindent a) encontrar la menor soluci\'on positiva de la ecuaci\'on $2x = \tan{(x)}$ con un error menor a $10^{-5}$. Cu\'antos pasos son necesarios si se comienza con el intervalo $[0.8,1.4]$?

\noindent b) encontrar una aproximaci\'on a $\sqrt{3}$ con un error menor a
 $10^{-5}$. Note que $\sqrt{3}$ es la ra�z positiva de la ecuaci�n $f(x)=x^2 -3$.



\bigskip

%2 
\prob{}
Desarrolle un programa para encontrar la ra�z de una funci�n
$f$ utilizando el m�todo de Newton--Raphson, dando como datos de entrada
una estimaci�n inicial $x_0$, la tolerancia \verb|tol| y un
n�mero m�ximo de iteraciones \verb|MAX_ITE|.
El programa debe finalizar cuando se satisfaga una de las siguientes
condiciones:
$$
\frac{|x_N - x_{N-1}|}{|x_N|} < \varepsilon, \qquad
|f(x_N)| < \varepsilon, \qquad
\mbox{N�mero de iteraciones} = \verb|MAX_ITE|
$$
El programa debe retornar (en pantalla) el n�mero de iteraciones realizadas, el valor final de
la aproximaci�n $x_N$, el error relativo estimado, y el valor de $|f(x_N)|$.  $f$ y
$f^{\prime}$ deben ser funciones del programa. Adem\'as, en un archivo, debe escribir
en cada iteraci\'on: $N$, $x_N$, $f(x_N)$ y $|x_N - x_{N-1}|/|x_N|$, con 13 cifras significativas para los reales.

Utilice este programa para resolver los incisos a) y b) del problema 1. 
Compare la cantidad de evaluaciones de la funci\'on y su derivada en 
los dos m\'etodos.

\bigskip

%10 
\prob{} Grafique el error relativo y el error relativo estimado ($|x_k - x_{k-1}|/|x_k|$) 
de la aproximaci\'on $k$-\'esima de $\sqrt{3}$  (pensada como ra\'iz de la funci\'on $f(x) = x^2-3$) 
en funci\'on de $k$, empleando el m\'etodo de bisecci\'on (use el intervalo inicial $[0.,2.5]$ y el 
de Newton--Raphson (use $x_0 = 2.5$). 
Escriba los programas en doble precisi\'on, y fije una tolerancia de $10^{-10}$ como
criterio de detenci\'on. Compare los resultados en un \'unico gr\'afico en 
escala {\em log -- log}. Cree un archivo {\em postscript color} con el gr\'afico.
Recuerde de poner t\'itulo, nombre a los ejes y leyendas para las distintas curvas.
\bigskip

\bigskip

%12 
%\prob{} Dado el siguiente polinomio
%\[
%p(x) = -10 + 5 x - 12 x^2  + 6 x^3  - 2 x^4  + x^5 \ .
%\]
%\noindent
%Grafique el mismo utilizando {\em gnuplot} y observe que posee 
%una \'unica ra\'iz real positiva, encuentre la misma utilizando:
%
%\begin{enumerate}
%\item El m\'etodo de bisecci\'on. Elija los valores iniciales utilizando
%los teoremas que acotan la regi\'on del espacio complejo donde se encuentran las ra\'ices.
%Eval\'ue el polinomio en una subrutina y utilice el algoritmo de Horner.
%
%\item El m\'etodo de Newton-Raphson. Elija el valor inicial utilizando
%los teoremas que acotan la regi\'on del espacio complejo donde se encuentran las ra\'ices.
%Eval\'ue el polinomio y su derivada en una subrutina utilizando el algoritmo de Horner.

%\end{enumerate}

%13
\prob{} Un objeto en ca\'{i}da vertical en el aire est\'a sujeto a la fuerza
de gravedad y a la resistencia del aire. Si un objeto de masa $m$ es
dejado caer desde una altura $h_0$, su altura luego de $t$ segundos est\'a
dada por:
$$
h(t) = h_0 - \frac{mg}{k} t + \frac{m^2 g}{k^2} \left( 1 - e^{-kt/m}\right)
$$
\noindent
donde $g=9.8\, m/s^2$ y $k$ representa el coeficiente de resistencia del aire
en $kg / s$.
Suponga que $h_0 = 10m$, $m=0.1\, kg$, y $k = 0.149\,  kg/s$.
Grafique $h(t)$ usando {\em gnuplot} para analizar su comportamiento.
Encuentre, con una precisi\'on de $0.01\,s$, el tiempo que le toma a este
objeto llegar al suelo. Utilice el m\'etodo de bisecci\'on y el de
Newton--Raphson.

%
\prob{} Resuelva num�\-ricamente, utilizando el m�todo de Newton-Raphson, 
el problema de encontrar la ra�z que satisfaga $x^3 - 2x - 5 = 0$.

Utilizando calculadora y siete cifras decimales calcule los primeros elementos de la secuencia,
comenzando con $x_0 = 2.000000$. Pare cuando el error absoluto en el eje de las abcsisas, 
definido como $\epsilon_x= |x_n - x_{n-1} |$ sea menor a $10^{-5}$.
Escriba en un papel a mano una tabla que en cada fila tenga $n$, $x_n$, $f(x_n)$ y $\epsilon_x$.

\bigskip

\bigskip

\noindent {\bf Ejercicios Complementarios}



\prob{}  Escriba un programa para hallar ls soluci\'on a la ecuaci\'on 
\[
x - \cos x = 0 
\]
\noindent
en el intervalo $[0, \pi/2]$.
\begin{enumerate}
\item utilizando el m\'etodo de la secante.

\item utilizando el m\'etodo de {\em Regula Falsi}.
\end{enumerate}

%11 
\prob{}
Adapte el programa de Newton--Raphson para calcular
una aproximaci\'on a la ra\'{\i}z c\'ubica de un n\'umero 
$R$ positivo. La entrada debe
ser el n\'umero $R$, la aproximaci\'on inicial $x_0$
y el error m\'aximo permitido $\varepsilon$.
 

%\prob{} Aplicar algoritmo de Newton-Raphson generalizado al siguiente sistema de 
%ecuaciones no lineales:
%\begin{eqnarray*}
%2x + \cos (y) &=& 0 \\
%2y + \sin (x) &=& 0
%\end{eqnarray*}
%\noindent
%Grafique primero las funciones, y encuentre la soluci\'on gr\'aficamente.
%Implemente un programa para resolver este sistema que tome como valor
%incial el par $(0,0)$, y escriba los valores sucesivos de los pares $(x_k,y_k)$
%obtenidos en la iteraci\'on $k$ y la estimaci\'on del error absoluto, definido commo
%$||{\bf e}_k||_\infty = || {\bf x}_k - {\bf x}_{k-1}||_\infty \equiv 
%{\tt max}(|x_k - x_{k-1}|,|y_k - y_{k-1}|)$. El programa debe detenerse cuando
%$||{\bf e}_k||_\infty < 10^{-6}$.



\end{document}
